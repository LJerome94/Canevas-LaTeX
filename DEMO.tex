\documentclass[12pt, a4paper]{article}

\usepackage{mise_en_page}
% Authors: Alexandre Simoneau, Jérôme Leblanc

% Insert using
%\input{Func.tex}

%\usepackage{mise_en_page}
\usepackage{amssymb}
\usepackage{amsmath}
\usepackage{amsfonts}
\usepackage[nice]{nicefrac}
\usepackage{braket}
\usepackage{mathrsfs}
\usepackage{lmodern}
\usepackage{bbm}
\usepackage{esint}
\usepackage{cancel}
\usepackage{subfiles}

%%%%%%%%%%%%%%%%%%%%%%%%%%%
% Unitées internationales %
%%%%%%%%%%%%%%%%%%%%%%%%%%%
\newcommand{\ampere}{\text{A}}
\newcommand{\bell}{\text{B}}
\newcommand{\celsius}{\degree\text{C}}
\newcommand{\coulomb}{\text{C}}
\newcommand{\degree}{\,^{\circ}}
\newcommand{\farad}{\text{F}}
\newcommand{\electro}{\text{e}}
\newcommand{\gram}{\text{g}}
\newcommand{\henry}{\text{H}}
\newcommand{\hertz}{\text{Hz}}
\newcommand{\hour}{\text{h}}
\newcommand{\joule}{\text{J}}
\newcommand{\kelvin}{\text{K}}
\newcommand{\m}{\text{m}}
\newcommand{\minute}{\text{m}}
\newcommand{\mole}{\text{mol}}
\newcommand{\newton}{\text{N}}
\newcommand{\ohm}{\Omega}
\newcommand{\pascal}{\text{Pa}}
\newcommand{\rad}{\text{rad}}
\newcommand{\second}{\text{s}}
\newcommand{\tesla}{\text{T}}
\newcommand{\torr}{\text{Torr}}
\newcommand{\volt}{\text{V}}
\newcommand{\watt}{\text{W}}
%
\newcommand{\tera}{\text{T}}
\newcommand{\giga}{\text{G}}
\newcommand{\mega}{~\text{M}}
\newcommand{\kilo}{~\text{k}}
\newcommand{\deci}{\text{d}}
\newcommand{\centi}{\text{c}}
\newcommand{\milli}{\text{m}}
\newcommand{\micro}{\mu}
\newcommand{\nano}{\text{n}}
\newcommand{\pico}{\text{p}}
\newcommand{\femto}{\text{f}}
%
\newcommand{\per}{\text{/}}
%
\newcommand{\Time}[3]{#1\hour~#2\minute~#3\second}
%\newcommand{\Angle}[3]{#1^{\circ}~#2'~#3''}

% Plus bel epsilon
\let\oldepsilon\epsilon
\let\epsilon\varepsilon
\let\varepsilon\oldepsilon

% Équations
\newcommand{\al}[1]{\begin{align} #1 \end{align}}
\newcommand{\eq}[1]{\begin{equation} \begin{aligned} #1 \end{aligned} \end{equation}}
\newcommand{\eqn}[1]{\eq{ #1 \notag }}
\newcommand{\sys}[1]{\begin{cases} #1 \end{cases}}

% Mathématiques de base
\newcommand{\Exp}[1]{\text{e}^{#1}}		% e^#
\newcommand{\p}[1]{\left( #1 \right)}	% (#)
\newcommand{\cro}[1]{\left[ #1 \right]}	% [#]
\newcommand{\norm}[1]{\left| #1\right|}	% |#|
\newcommand{\ee}[1]{\times 10^{#1}}		% X 10^#
\newcommand{\Expi}[1]{\Exp{i\p{#1}}}	% e^i(#)
\newcommand{\Expmi}[1]{\Exp{-i\p{#1}}}	% e^-i(#)
\newcommand{\avg}[1]{\left\langle #1 \right\rangle} % <#>
\newcommand{\acc}[1]{\left\lbrace #1 \right\rbrace} % {#}
\renewcommand{\hat}[1]{\widehat{#1}} % Remplacer le chapeau

% Vecteurs
\newcommand{\ve}[1]{\mathbf{#1}}
\newcommand{\vu}[1]{\hat{\ve{#1}}}
\newcommand{\vue}[1]{\vu{e}_{#1}}
\newcommand{\tens}{\otimes}
\newcommand{\sumdir}{\oplus}

% Matrices
\newcommand{\mat}[1]{\begin{bmatrix} #1 \end{bmatrix}}
\newcommand{\pmat}[1]{\begin{pmatrix} #1 \end{pmatrix}}
\newcommand{\deter}[1]{\norm{\begin{matrix} #1 \end{matrix}}}

\newcommand{\matnull}{\mathbb{O}} % Matrice Nulle

% Dérivées
\newcommand{\D}{\text{d}}
\newcommand{\dd}[2]{\frac{\D#1}{\D#2}}
\newcommand{\ddn}[3]{\frac{\D^{#1}#2}{\D #3^{#1}}}
\newcommand{\dx}[1]{\dd{#1}{x}}
\newcommand{\dy}[1]{\dd{#1}{y}}
\newcommand{\dt}[1]{\dd{#1}{t}}
\newcommand{\del}{\partial}
\newcommand{\ddp}[2]{\frac{\del#1}{\del#2}}
\newcommand{\ddpn}[3]{\frac{\del^{#1}#2}{\del #3^{#1}}}

\newcommand{\eval}[1]{\left. {#1} \right|}


% Opérateurs
\newcommand{\Det}[1]{\text{det}\p{#1}}
\newcommand{\Tr}[1]{\text{TR}\p{#1}}
\newcommand{\Sinc}[1]{\text{sinc}\p{#1}}
\newcommand{\Sin}[1]{\sin{\p{#1}}}% Rapports trigonométriques avec parenthèses automatiques
\newcommand{\Cos}[1]{\cos{\p{#1}}}
\newcommand{\Tan}[1]{\tan{\p{#1}}}
\newcommand{\Asin}[1]{\arcsin{\p{#1}}}
\newcommand{\Accos}[1]{\arcos{\p{#1}}}
\newcommand{\Atan}[1]{\arctan{\p{#1}}}
\newcommand{\fourier}[1]{\mathcal{F}\acc{#1}}
\renewcommand{\Re}[1]{\text{Re}\acc{#1}}
\renewcommand{\Im}[1]{\text{Im}\acc{#1}}

% Symboles
\newcommand{\N}{\mathbbm{N}}	% Naturels
\newcommand{\Z}{\mathbbm{Z}}	% Entiers
%\newcommand{\Q}{\mathbbm{Q}}	% Rationels
\newcommand{\R}{\mathbbm{R}}	% Réels
\renewcommand{\C}{\mathbbm{C}}	% Complexes

% Constantes et autres symboles physiques
\newcommand{\eo}{\epsilon_0}
\newcommand{\K}{\frac{1}{4\pi\eo}}
\newcommand{\lagrange}{\mathcal{L}}
\usepackage{references}
\usepackage{special}

%%%%%%%%%%%%%%
% Page titre %
%%%%%%%%%%%%%%
\title{Titre du travail}
\author{\JL} %Change pour votre ou vos noms
\teacher{Professeur}
\class{Titre du cours}
\date{\US\\ \today} % Changer ici pour un établissement ou date différente

\setlength{\parskip}{1em}

%%%%%%%%%%%%%%%%%%%%%%%%%
% Commencer le document %
%%%%%%%%%%%%%%%%%%%%%%%%%
\begin{document}

% Faire la page titre
\maketitlepage

%--------------------------------------------------------------------------------------------
\section{Ceci est une section}

\subsection{Ceci est une sous-section}

\subsubsection{Ceci est une sous-sous-section}

On peut écrire des équations sans numéros:
\eqn{
    E=mc^2
}
On peut aussi en écrire avec des numéros mais pas à toutes les lignes:
\al{
    F&=ma\label{eq:loi_de_newton}\\
    ma&=F\notag\\
    a&=\frac{F}{m}
}
Que l'on peut référer \eqref{eq:loi_de_newton}

On peut aussi faire des système d'équation
\eqn{
    \sys{
        x = 1\\
        y = 2\\
        z = 3
    }
}
On peut juxtaposer des équations (fonctinne seulement avec \textit{al\{\}}):
\al{
    \begin{split}
        Av=\lambda v
    \end{split}
    \begin{split}
        \ddot{\psi}=-\omega^2\psi
    \end{split}\notag
}

\bigbox{
On peut faire des section dans une boîte et faire des équation elles-aussi encadrées:
\eqn{
    \boxed{\psi(t)=A\cos{\omega t}+B\sin{\omega t}}
    }
}

On peut \yellow{aussi} \purple{mettre} \blue{du} \orange{texte} \green{en} \red{couleur}

On peut faire des listes:
\begin{itemize}
    \item premier
    \item deuxième
    \item troisième
\end{itemize}

Et des listes énumérées:
\begin{enumerate}
    \item premier
    \item deuxième
    \item troisième
\end{enumerate}

On a aussi le formatage automatique de plusieurs encadrés:
\eqn{
    \p{\frac{A}{B}} \cro{\frac{C}{D}} \norm{\frac{E}{F}} \avg{\frac{G}{H}} \acc{\frac{I}{J}}
}
Ce formatage s'applique aussi aux rapport trigonométriques:
\eqn{
    \Sin{\frac{A}{B}}~\Atan{\frac{B}{C}}~\Re{\frac{y}{x}}~\Im{z}
}

On peut écrire des vecteurs de divers façons:
\eqn{
    \Vec{x}~\ve{x}~\hat{x}~\vu{x}
}
On a certain raccourci pour les ensembles:
\eqn{
    \N~\Z~\R~\C
}
Sinon nous avons plusieurs polices:
\eqn{
    \mathbf{A B C D E F G H I J K L M N O P Q R S T U V W X Y Z}\\
    \mathcal{A B C D E F G H I J K L M N O P Q R S T U V W X Y Z}\\
    \mathbb{A B C D E F G H I J K L M N O P Q R S T U V W X Y Z}\\
    \mathbbm{A B C D E F G H I J K L M N O P Q R S T U V W X Y Z}\\
    \mathfrak{A B C D E F G H I J K L M N O P Q R S T U V W X Y Z}\\
    \mathscr{A B C D E F G H I J K L M N O P Q R S T U V W X Y Z}
}
On peut facilement mettre des figures comme suit:
% Les arguments sont {fichier image}{taille}{légende}{label}
\Figure{Demo/graphique.png}{0.5}{Ceci est une légende.}{fig:graphique}

On peut réfèrer aux figures (comme ceci \ref{fig:graphique}).

Les prochaines équations montres divers racourcis mathématiques:
\al{
    \begin{split}
        A=\mat{
        a & b \\
        c & d
        }=
        \pmat{
        a & b \\
        c & d
        }
    \end{split}
    \begin{split}
        \Det{A}=\deter{
        a & b \\
        c & d
        }
    \end{split}\notag
}
\eqn{
    (4\ee{12})~\Exp{\gamma}~\Expi{\omega t+\phi}~\Expmi{\omega t+\phi}\\
    \D~\dd{\phi}{r}~\ddn{3}{\phi}{r}~\dx{f}~\dy{f}~\dt{f}~\del~\ddp{f}{x}~\ddpn{3}{f}{y}~\eval{\dd{\phi}{r}}_{1}\\
    \int~\iint~\iiint~\oint~\oiint~\ointctrclockwise~\varointclockwise
}
Et pour les unités de mesures (ces lignes sont plus sensées si on voits le code):
\eqn{
    \ampere~\celsius~\coulomb~\degree~\gram~\henry~\hertz~\hour~\farad~\kelvin~\m~\joule~\minute~\mole~\newton~\ohm~\rad~\second~\volt~\watt\\
    \giga\m~\mega\m~\kilo\m~\deci\m~\centi\m~\milli\m~\micro\m~\nano\m~\pico\m\\
    \m\per\second\\
    \Time{2}{30}{46}
}

Ensuite des constantes, symboles ou opérateurs physique:
\eqn{
    \eo~\K~\fourier{f}~\lagrange
}

On peut annuler des termes:
\eqn{
    \xcancel{\Sin{\frac{Y+X}{Z}}}~~~\cancelto{1}{\Exp{\frac{D+X}{y}}}~~~\cancel{\frac{C}{AB}}
}

\partie{On peut avoir des titre surlignés}

% Importer le code d'un autre fichier
\subfile{Demo/sous_fichier}

%%%%%%%%%%%%%%%
% Situationel %
%%%%%%%%%%%%%%%
On peut présenter du code avec des sections comme suit (requiert une écriture particulière):
% Code
\begin{minted}{python}
import numpy as np

def func(x,y):
    if y == True:
        return 4*x+3
    
print("Hello world")
\end{minted}

On peut faire directement de graphiques dans \LaTeX{}:
% Graphique
\begin{figure}[H]
	\centering
	\begin{tikzpicture}
	\begin{axis}[
		my axis style,
		width=0.8\textwidth,
		height=.5\textwidth,
		legend entries={
			$y = x\Exp{-x}$,
			$y = 2x\Exp{-x}$,
			$y = \frac{x}{2}\Exp{-x}$
		},
		legend pos=north east
	]
	
	\addplot[
		domain=0:5,
		thick,
		-
	]
	{x*exp(-x)};

	\addplot[
		domain=0:5,
		thick,
		red,
		dashed,
		-
	]
	{2*x*exp(-x)};

	\addplot[
		domain=0:5,
		thick,
		blue,
		dashdotted,
		-
	]
	{.5*x*exp(-x)};
	
	\fill[
		black
	]
	(1,.36788) circle (2pt) node[above right] { $(1, 1/\Exp{})$};
	
	\end{axis}
	\end{tikzpicture}
	\caption{Ceci est un graphique.}
	\label{fig:my-awesome-graph}
\end{figure}

On peut faire toute sorte de diagrammes:
\al{
    \feynmandiagram [horizontal=a to b] { % Feynman
        i1 -- [fermion] a -- [fermion] i2,
        a -- [photon] b,
        f1 -- [fermion] b -- [fermion] f2,
    };
    \\
    \begin{circuitikz} \draw %Circuit électrique
    (0,0) to[battery] (0,4)
        to[ammeter] (4,4) -- (4,0) -- (3.5,0)
        to[lamp, *-*] (0.5,0) -- (0,0)
    (0.5,0) -- (0.5,-2)
        to[voltmeter] (3.5,-2) -- (3.5,0)
    ;
    \end{circuitikz}
    \\
    \begin{circuitikz} \draw % Circuit logique
    (0,2) node[and port] (myand1) {}
    (0,0) node[and port] (myand2) {}
    (2,1) node[xnor port] (myxnor) {}
    (myand1.out) -- (myxnor.in 1)
    (myand2.out) -- (myxnor.in 2);
    \end{circuitikz}
    \\
    \chemfig{A*6(-B=C(-CH_3)-D-E-F(=G)=)}
}

%Échecs
\newgame
\mainline{1.e4}    
\showboard


Plusieurs symboles sont disponibles:
\faAndroid \faAmazon\faAreaChart \faAssistiveListeningSystems \faBank \faBellSlashO \faCab \faChrome \faCut \faFilePdfO ~ne sont que quelques exemples.

Aussi des touches de calviers:
\eqn{
    \Return~\Spacebar~\Ctrl~\Shift~\keystroke{F1}~\keystroke{A}
}

%%%%%%%%%%%%%%%%%%%%%%%%
% Inutiles mais drôles %
%%%%%%%%%%%%%%%%%%%%%%%%
% Pour utiliser cette partie, enlever des commentaires les packages dans "mise_en_page.tex".
Maintenant pour le plaisir:
\eqn{
    \pumpkin~\skull~\mathbat~\mathghost~\mathcloud~\mathwitch~\bigskull~\bigpumpkin\\
    \fullmoon~\newmoon~\leftmoon~\rightmoon~\astrosun~\mercury~\earth~\jupiter~\saturn ~\uranus~\neptune~\pluto\\
    \aries~\taurus~\gemini~\cancer~\leo~\virgo~\libra~\scorpio~\sagittarius ~\capricornus~\aquarius~\pisces~\\
    \textpmhg{A B C D E F G H I J K L M N O P Q R S T U V W X Y Z 1 2 3 4 5 6 7 8 9 0}\\
}
\EOofficerI\EOSaw\EOPatron\EOScorpius\EORain\EOxvii \EOpi\EOki\EOSa

% Ajouter une page blanche
\blankpage

%--------------------------------------------------------------------------------------------
%%%%%%%%%%%%%%%%%%%%%%%%
% Terminer le document %
%%%%%%%%%%%%%%%%%%%%%%%%

\end{document}